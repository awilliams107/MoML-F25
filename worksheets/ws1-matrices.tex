\documentclass[12pt]{amsart}
\usepackage[margin=1in]{geometry}

\usepackage{tikz}

\usepackage{amsmath,amsfonts,amssymb,amsthm}
\usepackage{enumitem}
\usepackage{colonequals,multicol}
\usepackage{xcolor}
\usepackage{fancyhdr}
\usepackage{cleveref}

\newcommand{\Q}{\mathbb{Q}}
\newcommand{\N}{\mathbb{N}}
\newcommand{\Z}{\mathbb{Z}}
\newcommand{\R}{\mathbb{R}}
\DeclareMathOperator{\im}{image}
\DeclareMathOperator{\id}{id}


\theoremstyle{definition} 
\newtheorem*{definition}{Definition}
\newtheorem*{example}{Example}
\newtheorem{theorem}{Theorem}
\newtheorem{lemma}{Lemma}
\newtheorem{corollary}{Corollary}
\newtheorem{proposition}{Proposition}
\newtheorem{statement}{Statement}
\newtheorem*{remark}{Remark}



\begin{document}
	
\thispagestyle{fancy}
\pagestyle{fancy}
\lhead{\scriptsize University \textit{of} Nebraska-Lincoln}
\rhead{\scriptsize Math of Machine Learning}
\chead{Worksheet 1}
	
	\
 
\begin{center}
    {\Large \bf {\sc Vector and Matrix Algebra}}
\end{center}

%\section*{Vectors}
\begin{definition}
    A \textbf{scalar} is a quantity with size, often called magnitude, but no direction. 
\end{definition}

\noindent For example, the set of real numbers $\R$ are scalars!

\begin{definition}
    A \textbf{vector} is a quantity with size and direction.
\end{definition}

\begin{enumerate}[itemsep=0.8em,leftmargin=0pt]
%%%%%%%%%%%%%%%%%%%%%%%%%%%%%%%%%%%%%%%%

\item Draw examples of vectors in the Cartesian plane $\R^2$ and 3-dimensional $\R^3$.

\vspace{0.8em}
%%%%%%%%%%%%%%%%%%%%%%%%%%%%%%%%%%%%%%%%
\noindent The point that a vector starts at is often called the ``tail'' of the vector and where it ends is referred to as the ``head''.

\begin{definition}
    Let $n$ be a positive integer. The set of points $(x_1,x_2,\ldots,x_n)$ where each $x_1,x_2,\ldots,x_n$ are real numbers is an $n$-tuple of real numbers. We denote the set of all $n$-tuples of real numbers by $\R^n$.
\end{definition}

\noindent For example, $\R$ is the set of all $1$-tuples and $\R^2$ is the set of all 2-tuples or more familiarly, $xy$-pairs.
%%%%%%%%%%%%%%%%%%%%%%%%%%%%%%%%%%%%%%%%

\item Notice that vectors are independent of position because they only depend on size and direction. What would be a good way to distinguish between vectors? What do you notice if you were to put every vector's tail at the origin--the point $(0,0,\ldots,0)$ in $\R^n$?

\vspace{0.8em}
%%%%%%%%%%%%%%%%%%%%%%%%%%%%%%%%%%%%%%%%
\noindent \textbf{Note:} We often use a general subscript $n$, as in $\R^n$, when discussing a fact or property that holds for any real space $\R^n$.

\begin{definition}
    A vector is in \textbf{standard position} if its tail is on the origin.
\end{definition}

\begin{definition}
    Consider a vector $\mathbf{v}$ in standard position and suppose the head of the vector is at the point $(v_1,v_2,\ldots,v_n)$. The \textbf{component form} of $\mathbf{v}$ is 
    \[
        \left[\begin{array}{c}
            v_1 \\
            v_2 \\
            \vdots \\
            v_n
        \end{array}\right]
    \]
\end{definition}

\noindent \textbf{Note.} We are using ``column vector'' notation, but we can also write vectors as ``row vectors'' like $[v_1,v_2,\ldots,v_n]$.\\

\noindent \textbf{``Head$-$Tail'' Formula in $\mathbb{R}^n$.}
    Suppose a vector's tail is at a point $A=(a_1,a_2,\ldots,a_n)$ and head is at a point $B=(b_1,b_2,\ldots,b_n)$, then the coponent form of this vector $\vec{AB}$ is
    \[
        \vec{AB} =
        \left[\begin{array}{c}
             b_1 - a_1  \\
             b_2 - a_2 \\
             \vdots \\
             b_n-a_n
        \end{array}\right]
    \]

%%%%%%%%%%%%%%%%%%%%%%%%%%%%%%%%%%%%%%%%

\item Let $A=(0,-1,3,7)$ and $B=(2,1,-4,1)$. Compute $\vec{AB}$

\vspace{0.8em}
%%%%%%%%%%%%%%%%%%%%%%%%%%%%%%%%%%%%%%%%

\noindent \textbf{Notation.} We usually write vectors in bold like $\mathbf{v}$ or with an arrow over a letter like $\vec{v}$.

\noindent We can do lots of things with vectors! In particular, we can scale and add them:

\begin{definition}
    Let $\mathbf{u}=\begin{bmatrix}
    u_1\\
    u_2\\
    \vdots\\
    u_n
    \end{bmatrix}$ be a vector in $\R^n$, and let $k$ be a scalar, then
      $$k\mathbf{u}=k\begin{bmatrix}
    u_1\\
    u_2\\
    \vdots\\
    u_n
    \end{bmatrix}=\begin{bmatrix}
    ku_1\\
    ku_2\\
    \vdots\\
    ku_n
    \end{bmatrix}$$
\end{definition}

\begin{definition}
    Let $\mathbf{u}=\begin{bmatrix}
    u_1\\
    u_2\\
    \vdots\\
    u_n
    \end{bmatrix}$ and $\mathbf{v}=\begin{bmatrix}
    v_1\\
    v_2\\
    \vdots\\
    v_n
    \end{bmatrix}$ be vectors in $\R^n$.  We define $\mathbf{u}+\mathbf{v}$ by
      $$\mathbf{u}+\mathbf{v}=\begin{bmatrix}
    u_1\\
    u_2\\
    \vdots\\
    u_n
    \end{bmatrix}+\begin{bmatrix}
    v_1\\
    v_2\\
    \vdots\\
    v_n
    \end{bmatrix}=\begin{bmatrix}
    u_1+v_1\\
    u_2+v_2\\
    \vdots\\
    u_n+v_n
    \end{bmatrix}$$
\end{definition}

%%%%%%%%%%%%%%%%%%%%%%%%%%%%%%%%%%%%%%%%

\item Let $\mathbf{u}=\left[\begin{array}{c} 5\\-2\\1 \end{array}\right]$ and $\mathbf{v}=\left[\begin{array}{c} 0\\8\\4 \end{array}\right]$. Compute the following:
\begin{enumerate}
    \item $2\mathbf{u}$
    \item $\mathbf{u}+\mathbf{v}$
    \item $3\mathbf{u}-\frac{1}{2}\mathbf{v}$ 
\end{enumerate}

\vspace{0.8em}
%%%%%%%%%%%%%%%%%%%%%%%%%%%%%%%%%%%%%%%%

\begin{theorem}
    The following properties hold for vectors $\mathbf{u}$, $\mathbf{v}$ and ${\bf w}$ in $\R^n$ and scalars $k$ and $p$ in $\R$.
    \begin{enumerate}
      \item \label{item:commvectoradd} Commutative Property of Addition
      $$\mathbf{u}+\mathbf{v}=\mathbf{v}+\mathbf{u}$$
      \item \label{item:assocvectoradd}
      Associative Property of Addition
      $$(\mathbf{u}+\mathbf{v})+{\bf w}=\mathbf{u}+(\mathbf{v}+{\bf w})$$
      \item \label{item:identityvectoradd}
      Existence of Additive Identity: There exists a vector $\mathbf{0}$ such that
      $$\mathbf{u}+\mathbf{0}=\mathbf{u}$$
      \item \label{item:inversevectoradd}
      Existence of Additive Inverse: For every vector $\mathbf{u}$, there exists a vector $-\mathbf{u}$ such that
      $$\mathbf{u}+(-\mathbf{u})=\mathbf{0}$$
      \item\label{item:distvectoradd}
      Distributive Property over Vector Addition
      $$k(\mathbf{u}+\mathbf{v})=k\mathbf{u}+k\mathbf{v}$$
      \item\label{item:distvectoradd2}
      Distributive Property over Scalar Addition
      $$(k+p)\mathbf{u}=k\mathbf{u}+p\mathbf{u}$$
      \item \label{item:assocvectorscalarmult}
      Associative Property for Scalar Multiplication
      $$k(p\mathbf{u})=(kp)\mathbf{u}$$
      \item \label{item:onevectorscalarmult}
      Multiplication by 1
      $$1\mathbf{u}=\mathbf{u}$$
  \end{enumerate}
\end{theorem}

%%%%%%%%%%%%%%%%%%%%%%%%%%%%%%%%%%%%%%%%

\item Let $\mathbf{a}=\left[\begin{array}{c} 5\\-2 \end{array}\right]$, $\mathbf{b}=\left[\begin{array}{c} 0\\1 \end{array}\right]$, and $\mathbf{c}=\left[\begin{array}{c} 1\\1 \end{array}\right]$. Compute the following:
\begin{enumerate}
    \item $5(\mathbf{a}+2\mathbf{b}-\mathbf{c})$
    \item $3\mathbf{a}-7\mathbf{a}$
\end{enumerate}

\vspace{0.8em}
%%%%%%%%%%%%%%%%%%%%%%%%%%%%%%%%%%%%%%%%

\section*{Matrices}

\begin{definition}
     A \textbf{matrix} is a rectangular array of numbers. The plural form of matrix is \textbf{matrices}. Here is some vocabulary that will help us talk about matrices:
     \begin{itemize}
         \item The dimension of a matrix is defined as $m\times n$ where $m$ is the number of rows and $n$ is the number of columns.
         \item A \textbf{column vector} in $\mathbb{R}^m$ is an $m\times 1$ matrix.
         \item A \textbf{row vector} in $\mathbb{R}^n$ is an $1\times n$ matrix.
         \item We refer to the individual entries of the matrix by their position. The \textbf{$(i,j)$-entry} of a matrix is the entry in the $i^{\text{th}}$ row and $j^{\text{th}}$ column.
     \end{itemize}
\end{definition}

%%%%%%%%%%%%%%%%%%%%%%%%%%%%%%%%%%%%%%%%

\item Consider the matrix
$$M=\begin{bmatrix}
    1 & 2 & 3 & 4 \\
    5 & 2 & 8 & 7 \\
    6 & -9 & 1 & 2
\end{bmatrix}$$
\begin{enumerate}
    \item What are the dimensions of $M$?
    \item What is the second row of $M$?
    \item The entry $8$ is in which position?
    \item What value is in the $(3,2)$ position?
\end{enumerate}

\vspace{0.8em}
%%%%%%%%%%%%%%%%%%%%%%%%%%%%%%%%%%%%%%%%

\noindent \textbf{Notation.} We denote the entry in the $i^{th}$ row  and the $j^{th}$ column of an arbitrary $m\times n$ matrix $A$ by $a_{ij}$, and write $A$ in terms of its entries
as $$A= \begin{bmatrix} a_{ij} \end{bmatrix}=\begin{bmatrix}
           a_{11} & a_{12}&\dots&a_{1j}&\dots&a_{1n}\\
           a_{21}&a_{22} &\dots&a_{2j}&\dots &a_{2n}\\
		\vdots & \vdots&&\vdots&&\vdots\\
        a_{i1}&a_{i2}&\dots &a_{ij}&\dots &a_{in}\\
        \vdots & \vdots&&\vdots&&\vdots\\
		a_{m1}&a_{m2}&\dots &a_{mj}&\dots &a_{mn}
         \end{bmatrix}$$. 

Occasionally it will be convenient to talk about columns and rows of a matrix $A$ as vectors.  We will use the following notation:
    $$A=\begin{bmatrix}|&|&&|\\\mathbf{c}_1& \mathbf{c}_2 &\ldots & \mathbf{c}_n\\|&|&&|\end{bmatrix}\quad\text{or}\quad A=\begin{bmatrix}\mathbf{c}_1& \mathbf{c}_2 &\ldots & \mathbf{c}_n\end{bmatrix}$$
and
    $$A=\begin{bmatrix}
    - & \mathbf{r}_1 & - \\ - & \mathbf{r}_2 & - \\ & \vdots & \\ - & \mathbf{r}_m & -
    \end{bmatrix}\quad\text{or}\quad A=\begin{bmatrix}\mathbf{r}_1\\\mathbf{r}_2\\\vdots\\\mathbf{r}_m\end{bmatrix}$$

\begin{definition}[The Zero Matrix]\label{def:zeromatrix}
The $m\times n$ \textbf{zero matrix} is the $m\times n$ matrix
having every entry equal to zero. The zero matrix is
denoted by $O$.
\end{definition}

\begin{definition}[Equality of Matrices]\label{def:equalityofmatrices}
 Let $A=\begin{bmatrix} a_{ij}\end{bmatrix}$ and $B=\begin{bmatrix} b_{ij}\end{bmatrix}$ be two $m \times n$ matrices. Then $A=B$ means
that $a_{ij}=b_{ij}$ for all $1\leq i\leq m$ and 
$1\leq j\leq n$.
\end{definition}

\begin{definition}[Addition of Matrices]\label{def:additionofmatrices}
Let $A=\begin{bmatrix} a_{ij}\end{bmatrix} $ and $B=\begin{bmatrix} b_{ij}\end{bmatrix}$ be two
$m\times n$ matrices. Then the \textbf{sum of matrices} $A$ and $B$, denoted by $A+B$,  is an $m \times n$
matrix  given by 
$$A+B=\begin{bmatrix}a_{ij}+b_{ij}\end{bmatrix}$$

\end{definition}

%%%%%%%%%%%%%%%%%%%%%%%%%%%%%%%%%%%%%%%%

\item Find the sum of $A$ and $B$, if possible.
\begin{equation*}
A = \begin{bmatrix}
1 & 2 & 3 \\
1 & 0 & 4
\end{bmatrix}, ~
B = \begin{bmatrix}
5 & 2 & 3 \\
-6 & 2 & 1
\end{bmatrix}
\end{equation*}

\vspace{0.8em}
%%%%%%%%%%%%%%%%%%%%%%%%%%%%%%%%%%%%%%%%

\begin{definition}[Scalar Multiplication]\label{def:scalarmultofmatrices}
If $A=\begin{bmatrix} a_{ij}\end{bmatrix} $ and $k$ is a scalar,
then $kA=\begin{bmatrix} ka_{ij}\end{bmatrix}$. 
\end{definition}

%%%%%%%%%%%%%%%%%%%%%%%%%%%%%%%%%%%%%%%%

\item Find $7A$ if

$$A=\begin{bmatrix}
2 & 0 \\
1 & -4
\end{bmatrix}$$

\vspace{0.8em}
%%%%%%%%%%%%%%%%%%%%%%%%%%%%%%%%%%%%%%%%

\begin{theorem}[Properties of Matrix Addition]\label{th:propertiesofaddition}
Let $A,B$ and $C$ be matrices. Then, the following properties  hold. 

\begin{enumerate}
\item\label{item:mataddcomm} Commutative Law of Addition

$$A+B=B+A$$  


\item \label{item:mataddass} Associative Law of Addition

$$\left( A+B\right) +C=A+\left( B+C\right) $$


\item\label{item:mataddid} Additive Identity
\begin{center}
There exists a zero matrix such that
\end{center}
$$A+O=A$$


\item\label{item:mataddinv} Additive Inverse
\begin{center}
There exists a matrix, $-A$, such that
\end{center}
$$A+\left( -A\right) =O $$

\end{enumerate}
\end{theorem}

\begin{theorem}[Properties of Scalar Multiplication]\label{th:propertiesscalarmult}
Let $A, B$ be matrices, and $k, p$ be scalars. Then, the following properties properties of scalar multiplication hold.
\begin{enumerate}
\item\label{item:scalardistmatadd} Distributive Law over Matrix Addition
\begin{equation*}
k \left( A+B\right) =k A+ kB  
\end{equation*}

\item \label{item:matdistscalaradd}Distributive Law over Scalar Addition
\begin{equation*}
\left( k +p \right) A= k A+p A
\end{equation*}

\item \label{item:scalarmatmultass}Associative Law for Scalar Multiplication
\begin{equation*}
k \left( p A\right) = \left( k p \right) A 
\end{equation*}

\item\label{item:matmult1} Multiplication by $1$
\begin{equation*}
1A=A  
\end{equation*}
\end{enumerate}

\end{theorem}

%%%%%%%%%%%%%%%%%%%%%%%%%%%%%%%%%%%%%%%%

\item If 
$$A=\begin{bmatrix}2&-1\\3&4\end{bmatrix}\quad\text{and}\quad B=\begin{bmatrix}3&0\\-2&1\end{bmatrix}$$
then compute 
\begin{enumerate}
    \item 2(A+B)
    \item 3A-2B
\end{enumerate}

%%%%%%%%%%%%%%%%%%%%%%%%%%%%%%%%%%%%%%%%

\begin{definition} 
    Let $A$ be an $m \times n$ matrix with columns $\mathbf{a_{1}}, \cdots , \mathbf{a_{n}}$ and let $\mathbf{x}$ be a column vector in $\mathbb{R}^{n}$. We define the \textbf{product} of A and $\mathbf{x}$ as follows:
    
    \[A \mathbf{x} = 
    \begin{bmatrix}
    \mathbf{a_{1}} & \mathbf{a_{2}} & \cdots & \mathbf{a_{n}}\\
    \end{bmatrix}
     \begin{bmatrix}
    x_{1} \\
    x_{2} \\
    \vdots \\ 
    x_{n}
    \end{bmatrix}
    = x_{1}\mathbf{a_{1}}+ x_{2}\mathbf{a_{2}} + \cdots+ x_{n}\mathbf{a_{n}}
    \]
\end{definition}

%%%%%%%%%%%%%%%%%%%%%%%%%%%%%%%%%%%%%%%%

\item Determine which of the following products are defined. If the product is defined, compute it. Otherwise, explain why the product is undefined.
\begin{equation*}
\left [ \begin{matrix}
6 & 5 \\
2 & -3 \\
1 & 0 
\end{matrix} \right ]
\left[ \begin{matrix}
2 \\ -3
\end{matrix} \right]
\hspace{2.5in}
\left [ \begin{matrix}
-4 & 2 \\
1 & 6 \\
2 & -3 
\end{matrix} \right ]
\left[ \begin{matrix}
3 \\ -2 \\ 7
\end{matrix} \right]
\end{equation*}

%%%%%%%%%%%%%%%%%%%%%%%%%%%%%%%%%%%%%%%%

\item Compute $A\mathbf{x}$ if $$A=\begin{bmatrix}2&-1&3&2\\0&3&-2&1\\-2&4&1&0\end{bmatrix}\quad\text{and}\quad \vec{x}=\begin{bmatrix}3\\-1\\4\\1\end{bmatrix}$$

%\newpage
%%%%%%%%%%%%%%%%%%%%%%%%%%%%%%%%%%%%%%%%

\begin{definition}\label{def:dotproduct}
  Let $\vec{u}$ and $\vec{v}$ be vectors in $\R^n$.  The \textbf{dot
    product} of $\vec{u}$ and $\vec{v}$, denoted by
  $\vec{u}\cdot \vec{v}$, is given by
  \begin{align*}
    \vec{u}\cdot\vec{v}=\begin{bmatrix}u_1\\u_2\\\vdots\\u_n\end{bmatrix}\cdot\begin{bmatrix}v_1\\v_2\\\vdots\\v_n\end{bmatrix}=u_1v_1+u_2v_2+\ldots+u_nv_n.
  \end{align*}
\end{definition}

\newpage
%%%%%%%%%%%%%%%%%%%%%%%%%%%%%%%%%%%%%%%%

\item Compute the following dot products:
\begin{enumerate}[itemsep=0.5em]
    \item $\displaystyle \begin{bmatrix}3\\-1\\4\\1\\0\end{bmatrix} \cdot \begin{bmatrix}1\\5\\4\\7\\13\end{bmatrix}=$
    \item\label{ex:dotProdNotOrtho} $\displaystyle \begin{bmatrix}2\\1\end{bmatrix} \cdot \begin{bmatrix}3\\4 \end{bmatrix}=$
    \item\label{ex:dotProdOrtho} $\displaystyle \begin{bmatrix}2\\1\end{bmatrix} \cdot \begin{bmatrix}-1\\2 \end{bmatrix}=$
    \item $\displaystyle \begin{bmatrix}1\\0\end{bmatrix} \cdot \begin{bmatrix}0\\1 \end{bmatrix}=$
\end{enumerate}

%%%%%%%%%%%%%%%%%%%%%%%%%%%%%%%%%%%%%%%%

\item Sketch the vectors in parts~\ref{ex:dotProdNotOrtho} and~\ref{ex:dotProdOrtho}. Make a conjecture about when the dot product of two vectors is $0$.

\vspace{.8em}
%%%%%%%%%%%%%%%%%%%%%%%%%%%%%%%%%%%%%%%%

\begin{definition}\label{def:matmatproduct} Let $A$ be an $m\times n$ matrix whose rows are vectors $\vec{r}_1$, $\vec{r}_2,\ldots ,\vec{r}_n$.  Let $B$ be an $n\times p$ matrix with columns $\vec{b}_1, \vec{b}_2, \ldots, \vec{b}_p$.  Then the entries of the matrix product $AB$ are given by the dot products
\[AB=\begin{bmatrix}-&\vec{r}_1&-\\-&\vec{r}_2&-\\ &\vdots & \\-&\vec{r}_i &-\\ &\vdots& \\-&\vec{r}_m&-\end{bmatrix}
\begin{bmatrix}|&|&&|&&|\\\vec{b}_1& \vec{b}_2 &\ldots  & \vec{b}_j&\ldots& \vec{b}_p\\|&|&&|&&|\end{bmatrix}=
\begin{bmatrix}\vec{r}_1\cdot \vec{b}_1&\vec{r}_1\cdot \vec{b}_2&\ldots&\vec{r}_1\cdot \vec{b}_j&\ldots &\vec{r}_1\cdot \vec{b}_p\\\vec{r}_2\cdot \vec{b}_1&\vec{r}_2\cdot \vec{b}_2&\ldots&\vec{r}_2\cdot \vec{b}_j&\ldots &\vec{r}_2\cdot \vec{b}_p\\\vdots&\vdots&&\vdots&&\vdots\\\vec{r}_i\cdot \vec{b}_1&\vec{r}_i\cdot \vec{b}_2&\ldots&\vec{r}_i\cdot \vec{b}_j&\ldots &\vec{r}_i\cdot \vec{b}_p\\\vdots&\vdots&&\vdots&&\vdots\\\vec{r}_m\cdot \vec{b}_1&\vec{r}_m\cdot \vec{b}_2&\ldots&\vec{r}_m\cdot \vec{b}_j&\ldots &\vec{r}_m\cdot \vec{b}_p
\end{bmatrix}
\]
\end{definition}

%%%%%%%%%%%%%%%%%%%%%%%%%%%%%%%%%%%%%%%%

\item Suppose $A$ is any $3\times 2$-matrix, $B$ is a $2\times 2$-matrix and $C$ is a $2\times 3$-matrix. Which of the products $AB$, $BC$ and $CA$ are defined? What is the general rule for the product of two matrices to be defined?

%%%%%%%%%%%%%%%%%%%%%%%%%%%%%%%%%%%%%%%%

\item Compute the matrix product
$
AB = \left [  \begin{matrix} 
1 & 1 \\
1 & -1
\end{matrix} \right ] \left [ \begin{matrix}
-1 & -1 & 2 & 1 \\
-1 & 1 & 0 & 1
\end{matrix} \right ]
$

%%%%%%%%%%%%%%%%%%%%%%%%%%%%%%%%%%%%%%%%


\item If $a$ and $b$ are real numbers and we know $ab=0$, then either $a=0$ or $b=0$ (or both). This is not true for matrices. Give an example of two $2\times 2$-matrices $A$ and $B$ such that $AB=0$. %You will only need to make matrices with values $0$ and $1$.

%%%%%%%%%%%%%%%%%%%%%%%%%%%%%%%%%%%%%%%%

\item Provide an example of two $3\times 3$ matrices $A$ and $B$ such that $AB$ does not equal $BA$.

%%%%%%%%%%%%%%%%%%%%%%%%%%%%%%%%%%%%%%%%

\end{enumerate}


\end{document}